The differentiation of haemorrhagic from ischaemic stroke using Electrical Impedance Tomography (EIT) requires measurements at multiple frequencies, since the general lack of healthy measurements on the same patient excludes time-difference imaging methods. It has previously been shown that the inaccurate modelling of electrodes constitutes one of the largest sources of image artefacts in non-linear multi-frequency EIT applications. To address this issue, we augmented the conductivity Jacobian matrix with a Jacobian matrix with respect to electrode movement. Using this new algorithm, simulated ischaemic and haemorrhagic strokes in a realistic head model were reconstructed for varying degrees of electrode position errors. The simultaneous recovery of conductivity spectra and electrode positions removed most artefacts caused by inaccurately modelled electrodes. Reconstructions were stable for electrode position errors of up to \SI{1.5}{\milli\metre} standard deviation along both surface dimensions. We conclude that this method can be used for electrode model correction in multi-frequency EIT.


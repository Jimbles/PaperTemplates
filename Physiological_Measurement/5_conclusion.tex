Simultaneous iterative electrode position correction with the fraction reconstruction method using spectral constraints was applied to a numerical head phantom with realistic conductivities. Realistic noise was added to the simulated voltages to investigate the robustness of the proposed method. The results show that

\begin{enumerate}
\item the simultaneous recovery of tissue volume fractions and electrode position errors removed most image artefacts caused by inaccurately modelled electrodes.
\item while haemorrhagic strokes could be reconstructed with electrode position errors up to \SI{2}{\milli\metre} standard deviation in both surface dimensions, the reconstruction of ischaemic strokes was less reliable from electrode movements of \SI{1}{\milli\metre} onwards.
\end{enumerate}

Further work is required to understand why ischaemic stroke reconstructions were less reliable with the proposed method, and to correct for it. Additionally, it has so far not been studied how stable non-linear multi-frequency reconstruction methods are in the presence of geosdasdasdmetric modelling errors, such as skull shape. We plan to validate the presented results in tank experiments with 3D printed head shaped tanks and skull \citep{Jehl2015b} and recommend the presented algorithm for any planned human studies.
\subsection{Background}
Multifrequency Electrical Impedance Tomography (MFEIT) is a method for imaging biological tissues with frequency-dependent conductivity. While time-difference (TD) EIT is used to image conductivity changes between a baseline voltage measurement and a data measurement, MFEIT differentiates tissues based on their conductivity spectra at different modulation frequencies of the applied current. Therefore MFEIT does not require a baseline measurement and can be used as a diagnostic tool for multiple applications \citep{Brown1995, Hampshire1995, Malich2003}, including stroke type differentiation \citep{Holder1992,Romsauerova2006, Packham2012}.

While cerebral haemorrhages require surgery, ischaemic strokes could be treated with a clot dissolving drug if diagnosed within three hours of onset. The current diagnostic procedure is to take a CT scan, and results in only 2.5-6\% of the 80\% of ischaemic strokes to be treated in time \citep{Power2004, Saver2013}. While EIT cannot compete with CT in terms of image quality, the small size and low cost of an EIT system make it feasible to equip ambulances for early ischaemia diagnosis and thrombolytic treatment.

A novel non-linear method for performing MFEIT using spectral constraints was recently proposed by \citet{Malone2013}, because the complicated geometry of the head excludes linear multi-frequency reconstruction algorithms, such as weighted frequency difference \citep{Jun2009}. In the new MFEIT algorithm, the inverse problem is reformulated to express the conductivity at each frequency in terms of tissue volume fractions and known tissue conductivity spectra. This separation of variables makes it possible to use measurements at different frequencies simultaneously, since the tissue fractions are frequency independent. An analysis of the influence of modelling errors on images reconstructed with this method has shown that inaccurately modelled electrode positions strongly affect the image quality, whereas wrongly modelled contact impedances were negligible and spectral errors were tolerable as long as tissue spectra did not overlap \citep{Malone2014}.

Methods for correction of electrode modelling inaccuracies are recently gaining interest in the EIT community \citep{Soleimani2006, Darde2012}, and have already been applied to a realistic three dimensional head model for simultaneous TD recovery of conductivity changes and electrode movement \citep{Jehl2015b}. In this paper, the first application of simultaneous MFEIT image reconstruction and electrode model correction is demonstrated on a realistic 3D head model with skull and scalp. The implementation is discussed and the performance is evaluated for different levels of electrode position errors.

\subsection{Purpose}
The purpose of this study is to evaluate the performance of simultaneous recovery of electrode positions and conductivity spectrum changes. Specifically, the following two questions will be addressed:
\begin{enumerate}
\item Does the simultaneous recovery of conductivity changes and electrode modelling errors remove image artefacts caused by inaccurately modelled electrode positions?
\item At which magnitude of electrode position errors does the proposed algorithm begin to fail?
\end{enumerate}
To answer these questions, multi-frequency boundary voltages were simulated on a fine 5 million element head mesh with different levels of electrode position errors. Reconstructions were made with and without the proposed addition of electrode modelling corrections and the resulting images were compared. It was found that the proposed algorithm could stably recover simulated strokes in the presence of electrode modelling errors of up to \SI{1.5}{\milli\metre} standard deviation.
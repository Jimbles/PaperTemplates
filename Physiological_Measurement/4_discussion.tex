The simultaneous recovery of stroke tissue fractions and electrode positions significantly improved image quality in the presence of electrode modelling errors and had only small negative effects on the image quality when electrode were modelled accurately. For movements between \SI{0.5}{\milli\metre} and \SI{2}{\milli\metre} standard deviation along both surface dimensions, the average image error was 23\% compared to 55\% without electrode correction. While reconstructions of haemorrhagic strokes were even successful in the presence of \SI{2}{\milli\metre} of electrode errors, ischaemia detection was less reliable from \SI{1}{\milli\metre} onwards. The lower reliability could be caused by a worse differentiability of ischaemic changes to electrode movements, but this could so far not be confirmed.

Ideally, several reconstructions would have been made for each electrode movement level in order to characterise the effect of electrode modelling errors over a number of samples. However, the computational expense of multiple repetitions was prohibitive, since reconstruction of a single image took around 6 hours. For the same reason, only two stroke positions were studied. Nonetheless, the produced images clearly illustrate the advantage of simultaneous tissue fraction and electrode position recovery in the presence of electrode modelling errors.

Many methods for computing the Jacobian matrix with respect to electrode movement could have been used on the multi-frequency data. The presented method for correction of inaccurate electrode modelling had the advantage over previous approaches, that the mesh geometry did not have to be altered. This allowed for the iterative recovery of movements of several millimetres on fine meshes, which is not possible with traditional differential approaches for electrode movement recovery \citep{Soleimani2006}. Furthermore, the implementation based on the Fr\'{e}chet derivative \citep{Darde2012} used here has been shown to be fast and memory efficient \citep{Jehl2015b}.
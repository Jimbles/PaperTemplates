\subsection{Making a thing}

lalalala 
\subsection{Data simulation}

The simulation parameters in this study were identical to the ones used in the preliminary feasibility study \citep{Malone2014}. Boundary voltages were computed on a fine 5 million element mesh, which was created from a CT scan of a human head and included three homogeneous tissues: brain, skull and scalp. 


%
%\begin{figure*}[t!]	
%\centering
%	\subfloat[]{\centering\includegraphics[width=0.48\textwidth,trim=0.in 0.in 0.0in 0in, clip=true]{./Images/spectra.eps}
%		\label{spectra}}
%	\hfil
%	\subfloat[]{\centering\raisebox{0.5cm}{\includegraphics[width=0.24\textwidth,trim=5.5in 1in 4.5in 0.5in, clip=true]{./Images/stroke_side.eps}}
%	\label{figure_methods_model_side}}
%	\hfil
%	\subfloat[]{\centering\raisebox{0.5cm}{\includegraphics[width=0.24\textwidth,trim=6.7in 1.6in 4.5in .5in, clip=true]{./Images/stroke_back.eps}}
%	\label{figure_methods_model_back}}
%\caption{Model: (a) conductivity spectra of the simulated tissues, (b) simulated lateral and (c) posterior stroke position. These simulation parameters were already used in \citet{Malone2014}. \href{https://creativecommons.org/licenses/by/3.0/}{CC BY}}
%\end{figure*}


\subsection{things}
